%%
%% This is file `scsnowman-sample.tex', part of scsnowman package.
%% Maintained on GitHub:
%% https://github.com/aminophen/scsnowman
%%
%% Copyright (c) 2015-2016 Hironobu Yamashita
%%   Email   :  h.y.acetaminophen[a t]gmail.com
%%   GitHub  :  https://github.com/aminophen
%%   Blog    :  http://acetaminophen.hatenablog.com/
%%   Twitter :  @aminophen
%%
% platex + dvipdfmx
\documentclass[dvipdfmx]{jsarticle}
\usepackage[margin=27truemm]{geometry}
\usepackage{scsnowman}
\title{\textsf{scsnowman}パッケージの実用例}
\author{アセトアミノフェン}
\begin{document}
\maketitle

% ふつうのゆきだるま
これはゆきだるま\scsnowman です。

% 雪ありゆきだるま
今日の天気は\scsnowman[snow]です。

% 帽子をかぶったゆきだるま
ゆきだるま\scsnowman[hat]が帽子をかぶりました。

% 帽子をかぶったゆきだるま(帽子の色は青)
私は青い帽子をかぶった\scsnowman[hat=blue]が大好きです。

% 帽子とマフラー付(マフラーの色は赤)
マフラー\scsnowman[hat=true,muffler=red]を付けてあげましょう。

% 腕あり
腕も作って\scsnowman[hat=true,muffler=red,arms=true]あげましょう。

% サイズ変更
小\scsnowman、
中\scsnowman[scale=3]、
大\scsnowman[scale=5]。

% ゆきだるま三兄弟
\begin{table}[htb]
  \begin{tabular}{ccc}
    \texttt{U+2603} & \texttt{U+26C4} & \texttt{U+26C7} \\
    \texttt{SNOWMAN} & \texttt{SNOWMAN WITHOUT SNOW} & \texttt{BLACK SNOWMAN} \\
    \scsnowman[scale=5,body=false,snow=true] & \scsnowman[scale=5,body=false,snow=false] & \scsnowman[scale=5,body=true,snow=true]
  \end{tabular}
\end{table}

\begin{table}[htb]
  \begin{tabular}{ccc}
    \texttt{U+2603} & \texttt{U+26C4} & \texttt{U+26C7} \\
    \texttt{SNOWMAN} & \texttt{SNOWMAN WITHOUT SNOW} & \texttt{BLACK SNOWMAN} \\
    \scsnowman[scale=5,body=false,snow=true,muffler=true] & \scsnowman[scale=5,body=false,snow=false,muffler=true] & \scsnowman[scale=5,body=true,snow=true,muffler=true]
  \end{tabular}
\end{table}

\begin{table}[htb]
  \begin{tabular}{ccc}
    \texttt{U+2603} & \texttt{U+26C4} & \texttt{U+26C7} \\
    \texttt{SNOWMAN} & \texttt{SNOWMAN WITHOUT SNOW} & \texttt{BLACK SNOWMAN} \\
    \scsnowman[scale=5,body=false,snow=true,hat=true,arms=true] & \scsnowman[scale=5,body=false,snow=false,hat=true,arms=true] & \scsnowman[scale=5,body=true,snow=true,hat=true,arms=true]
  \end{tabular}
\end{table}

\begin{table}[htb]
  \begin{tabular}{ccc}
    \texttt{U+????} & \texttt{U+????} & \texttt{U+????} \\
    \texttt{RED SNOWMAN} & \texttt{BLUE SNOWMAN} & \texttt{GREEN SNOWMAN} \\
    \scsnowman[scale=5,body=red,snow=red,hat=red] & \scsnowman[scale=5,body=blue,hat=blue,arms=blue,muffler=blue] & \scsnowman[scale=5,body=green,snow=green,arms=green]
  \end{tabular}
\end{table}

\end{document}
